\chapter{函数逼近与计算}

\section{引言}

\subsection{函数逼近的问题的一般提法}

对于函数类$A$中给定的函数$f(x)$, 要求在另一类较简单且便于计算的函数类$B(\subset A)$中寻找一个函数$P(x)$, 使得函数$P(x)$与$f(x)$之差在某种度量意义下最小.

本章所研究的函数类$A$通常为区间$[a,b]$上的连续函数, 记作$C[a,b]$; 函数类$B$通常为\emph{代数多项式}或三角多项式.

\subsection{常用的度量标准}

\subsubsection{最佳一致逼近}

\begin{definition}[最佳一致逼近]
    若以函数$f(x)$和$P(x)$的最大误差
    \begin{equation*}
        \max_{x\in[a,b]}\abs{f(x)-P(x)}=\norm{f(x)-P(x)}_\infty
    \end{equation*}
    作为度量误差$f(x)-P(x)$"大小"的标准, 在这种意义下的函数逼近称为\emph{最佳一致逼近}或\emph{均匀逼近}
\end{definition}

\subsubsection{最佳平方逼近}

\begin{definition}[最佳平方逼近]
    采用
    \begin{equation*}
        \sqrt{\int_a^b\left[f(x)-P(x)\right]^2\dd{x}}=\norm{f(x)-P(x)}_2
    \end{equation*}
    作为度量误差"大小"标准的函数逼近称为\emph{最佳平方逼近}或\emph{均方逼近}.
\end{definition}