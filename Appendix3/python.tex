\chapter{复习题}

\section*{插值法}
1. 给出$f(x)=\ln x$的数值表

\begin{table}[h!]
    \begin{tabular}{c|ccccc}
        \hline
        $x$&0.4&0.5&0.6&0.7&0.8\\
        \hline
        $\ln x$&-0.916291&-0.693147&-0.510826&-0.357765&-0.223144\\
        \hline
    \end{tabular}
\end{table}

分别使用Lagrange插值法和Newton插值法, 构造线性插值和二次插值, 计算$\ln0.54$的近似值. 并估计其误差

2. 给定下面数值表
\begin{table}[h!]
    \begin{tabular}{c|ccccc}
        \hline
        $x$&1&-1&2\\
        \hline
        $f(x)$&0&3&4\\
        \hline
    \end{tabular}
\end{table}

2.1. 使用Newton插值法, 构造满足上述条件的插值多项式($P_2(x)$)

2.2. 在满足上面数值表的前提下, 若同时满足$f'(1)=1$, 试通过构造重节点差商表, 构造插值多项式($P_4(x)$)

\section*{函数逼近与拟合}

1. 对于函数$f(x)=e^x$, 

1.1. 求其在$[-1,1]$上最佳一次逼近多项式, 并估计其误差

1.2. 求其在$[-1,1]$上在$\spn{1,x}$上的最佳平方逼近多项式, 并估计其误差

1.3. 使用Legendre多项式做正交多项式, 求其在$[-1,1]$上的最佳平方逼近多项式, 并估计其误差

2. 设$f(x)=x^4+3x^3-1$, 通过Chebyshev多项式, 在$[0,1]$区间构造其三次最佳逼近多项式, 并估计其误差限.

3. 对于下面的数据

\begin{table}[h!]
    \begin{tabular}{c|ccccc}
        \hline
        $x_i$&19&25&31&38&44\\
        \hline
        $y_i$&19.0&32.3&49.0&73.3&97.8\\
        \hline
    \end{tabular}
\end{table}

使用最小二乘法, 完成下面的拟合

3.1. 使用$y=ax$的形式

3.2. 使用$y=a+bx^2$的形式

3.3. 使用$y=a+bx+cx^2$的形式

\section*{矩阵分析基础}

设矩阵
\begin{equation*}
    A=\mqty(
        6&5\\
        3&1
    )
\end{equation*}

1. 计算$A$行范数, 列范数, 2范数和F范数

2. 计算矩阵$A$的谱半径

3. 计算矩阵$A$的1-条件数,2-条件数和谱条件数

\section*{线性方程组的直接解法}

1. 用Gauss-Jordan方法求矩阵$A$的逆. 其中
\begin{equation*}
    A=\mqty(2&1&-3&-1\\
    3&1&0&7\\
    -1&2&4&-2\\
    1&0&-1&5)
\end{equation*}

2. 对线性方程组
\begin{equation*}
    \begin{cases}
        2x_1-x_2+x_3=4\\
        -x_1-2x_2+3x_3=5\\
        x_1+3x_2+x_3=6
    \end{cases}
\end{equation*}

使用下列方法求解方程组:

2.1. Gauss消去法

2.2. 列主元Gauss消去法

2.3. LU三角分解法

2.4. 紧凑格式列主元Doolittle三角分解法

2.5. 改进平方根法

\section*{线性方程组的迭代解法}

设方程组
\begin{equation*}
    \begin{cases}
        x_1-\frac{1}{4}x_3-\frac{1}{4}x_4=\frac{1}{2}\\
        x_2-\frac{1}{4}x_3-\frac{1}{4}x_4=\frac{1}{2}\\
        -\frac{1}{4}x_1-\frac{1}{4}x_2+x_3=\frac{1}{2}\\
        -\frac{1}{4}x_1-\frac{1}{4}x_2+x_4=\frac{1}{2}
    \end{cases}
\end{equation*}

求解该方程组的Jacobi迭代矩阵和Gauss-Seidel迭代矩阵, 并判断其收敛性.