\chapter{线性方程组的直接解法}

\section{Gauss消去法}

\subsection{三角形方程组回代法}

如:
\begin{equation*}
    \begin{cases}
        a_{11}x_1&=b_1\\
        a_{21}x_1+a_{22}x_2&=b_2\\
        \cdots\quad\cdots\quad\cdots\\
        a_{n1}x_1+a_{n2}x_2+\cdots+a_{nn}x_n&=b_n
    \end{cases}
\end{equation*}

或上三角方程组:
\begin{equation*}
    \left\{
    \begin{aligned}
        a_{11}x_1+a_{12}x_2+\cdots+a_{1n}x_n=b_1\\
        a_{22}x_2+\cdots+a_{2n}x_n=b_2\\
        \cdots\qquad\qquad\qquad\\
        a_{nn}x_n=b_n
    \end{aligned}
    \right.
\end{equation*}

\subsection{顺序Gauss消去法}

对于一般的线性方程组
\begin{equation*}
    \begin{cases}
        a_{11}x_1+a_{12}x_2+a_{13}x_3+\cdots+a_{1n}x_n=a_{1,n+1}\\
        a_{21}x_1+a_{22}x_2+a_{23}x_3+\cdots+a_{2n}x_n=a_{2,n+1}\\
        \cdots\\
        a_{n1}x_1+a_{n2}x_2+a_{n3}x_3+\cdots+a_{nn}x_n=a_{n,n+1}\\
    \end{cases}
\end{equation*}

记
\begin{align*}
    \overline{A}^{(1)}&=(A,b)\\
    &=\mqty(
        a_{11}&a_{12}&\cdots&a_{1n}&a_{1,n+1}\\
        a_{21}&a_{22}&\cdots&a_{2n}&a_{2,n+1}\\
        \vdots&\vdots&&\vdots&\vdots\\
        a_{n1}&a_{n2}&\cdots&a_{nn}&a_{n,n+1}
    )=\mqty(
        a_{11}^{(1)}&a_{12}^{(1)}&\cdots&a_{1n}^{(1)}&a_{1,n+1}^{(1)}\\
        a_{21}^{(1)}&a_{22}^{(1)}&\cdots&a_{2n}^{(1)}&a_{2,n+1}^{(1)}\\
        \vdots&\vdots&&\vdots&\vdots\\
        a_{n1}^{(1)}&a_{n2}^{(1)}&\cdots&a_{nn}^{(1)}&a_{n,n+1}^{(1)}
    )
\end{align*}

解$n$阶方程组的顺序Gauss消去一般步骤

第一次消元, 设$a_{11}^{(1)}\ne0$, 令
\begin{equation*}
    m_{i1}=\frac{a_{i1}^{(1)}}{a_{11}^{(1)}}
\end{equation*}
称$m_{i1}$为\emph{消元因子}. 令$a_{ij}^{(2)}=a_{ij}^{(1)}-m_{i1}a_{1j}^{(1)}, i=2,\cdots,n, j=2,\cdots,n+1$, 则得矩阵
\begin{equation*}
    \mqty(
        a_{11}^{(1)}&a_{12}^{(1)}&\cdots&a_{1n}^{(1)}\\
        0&a_{22}^{(2)}&\cdots&a_{2n}^{(2)}\\
        \vdots&\vdots&&\vdots\\
        0&a_{n2}^{(2)}&\cdots&a_{nn}^{(2)}
    )\mqty(x_1\\x_2\\\vdots\\x_n)=\mqty(a_{1,n+1}^{(1)}\\a_{2,n+1}^{(2)}\\\vdots\\a_{n,n+1}^{(2)})
\end{equation*}
即$A^{(2)}x=b^{(2)}$

以此类推, 对于第$k$次消元, 若$a_{kk}^{(k)}\ne0$, 令
\begin{equation*}
    m_{ik}=\frac{a_{ik}^{(k)}}{a_{kk}^{(k)}}
\end{equation*}
令$a_{ik}^{(k+1)}=a_{ij}^{(k)}-m_{ik}a_{kj}^{(k)}, i=k+1,\cdots,n; j=k+1,\cdots,n+1$
则
\begin{equation*}
    A^{(k)}=\mqty(
        a_{11}^{(1)}&a_{12}^{(1)}&\cdots&\cdots&\cdots&a_{1n}^{(1)}\\
        &a_{22}^{(2)}&\cdots&\cdots&\cdots&a_{2n}^{(2)}\\
        &&\ddots&\vdots&\vdots&\vdots\\
        &&&a_{kk}^{(k)}&\cdots&a_{kn}^{(k)}\\
        &&&\vdots&&\vdots\\
        &&&a_{nk}^{(k)}&\cdots&a_{nn}^{(k)}
    ), b^{(k)}=\mqty(
        a_{1,n+1}^{(1)}\\a_{2,n+1}^{(2)}\\\vdots\\a_{k-1,n+1}^{(k-1)}\\a_{k,n+1}^{(k)}\\\vdots\\a_{n,n+1}^{(k)}
    )
\end{equation*}
即$A^{(k)}x=b^{(k)}$

共进行$n-1$步, 即可得到三角方程组
\begin{equation*}
    \mqty(
        a_{11}^{(1)}&a_{12}^{(1)}&\cdots&a_{1n}^{(1)}\\
        &a_{22}^{(2)}&\cdots&a_{2n}^{(2)}\\
        &&\ddots&\vdots\\
        &&&a_{nn}^{(n)}
    )\mqty(x_1\\x_2\\\vdots\\x_n)=\mqty(a_{1,n+1}^{(1)}\\a_{2,n+1}^{(2)}\\\vdots\\a_{n,n+1}^{(n)})
\end{equation*}
得
\begin{equation*}
    A^{(n)}x=b^{(n)}
\end{equation*}

利用回代过程, 求解得
\begin{align*}
    x_n=\frac{a_{n,n+1}^{(n)}}{a_{nn}^{(n)}}\\
    x_k=\frac{a_{k,n+1}^{(k)}-\sum_{j=k+1}^na_{kj}^{(k)}x_j}{a_{kk}^{(k)}}, k=n-1,n-2,\cdots,2,1
\end{align*}
称$a_{kk}^{(k)}$为\emph{主元素}(或\emph{主元}).

注: 若第一个主元素为0, 则可把第一列得非零元素调整至第一行, 此时Gauss消去法仍然可用.

一般地, 关于顺序Gauss消去法可执行的前提, 有如下引理:

\begin{lemma}
    给定线性方程组$Ax=b$, 按顺序Gauss消去法所形成的各主元素$a_{kk}^{(k)}(k=1,2,\cdots,n)$均不为零, 从而Gauss消去法可顺利执行的充要条件是$n$阶方阵$A$的所有顺序主子式都不为零, 即
    \begin{equation*}
        D_1=a_{11}\ne0,\cdots, D_k=\mqty|
        a_{11}&\cdots&a_{1k}\\
        \vdots&\ddots&\vdots\\
        a_{k1}&\cdots&a_{kk}|\ne0, k=2,3,\cdots,n
    \end{equation*}
\end{lemma}

当线性方程组的系数矩阵为\emph{对称正定}或\emph{严格对角占优}时, 按顺序Gauss消去法计算是稳定的.

\begin{theorem}
    若矩阵$A$非奇异, 即$\inv{A}$存在, 则可通过逐次消元以及行交换, 将方程组转化为三角形方程组并求出唯一解. 
\end{theorem}

\begin{theorem}
    若$A$的所有\emph{顺序主子式}均不为0, 则Gauss消元无需换行即可进行到底并求出唯一解.
\end{theorem}

\subsection{Gauss主元素消去法}

考虑一个方程组:
\begin{equation*}
    \mqty(
        0.001&2.000&3.000\\
        -1.000&3.712&4.623\\
        -2.000&1.072&5.643
    )\mqty(x_1\\x_2\\x_3)=\mqty(
        1.000\\2.000\\3.000
    )
\end{equation*}
可以发现, 该方程组在使用四位浮点数运算, 得到的计算解(近似解)为$\transpose{(-0.4000,-0.09980,0.4000)}$, 显然与精确解不符. 出现该问题的原因是, 在消元计算时, 使用了小主元0.001, 从顺序Gauss消去法的角度看, 不为零的主元可以使用, 但从误差的角度看, 使用小主元(做除数)会导致数量级增大, 从而引起舍入误差. 因此, 我们需要通过矩阵变换, 以获得合适的主元. 

\subsubsection{完全主元素法}

考虑在每一次消元过程中, 在增广矩阵中选择绝对值最大的一项作为主元素, 并将其移动至对角主元位置, 然后进行消元. 经过一系列的消元, 可将方程组化为
\begin{equation*}
    \mqty(
        a_{11}&a_{12}&\cdots&a_{1n}\\
        &a_{22}&\cdots&a_{2n}\\
        &&\ddots&\vdots\\
        &&&a_{nn}
    )\mqty(y_1\\y_2\\\vdots\\y_n)=\mqty(b_1\\b_2\\\vdots\\b_n)
\end{equation*}

其中, $y_1,y_2,\cdots,y_n$为未知数$x_1,x_2,\cdots,x_n$经过选择主元调换后的次序. 这里未知数次序的调换主要是由于列变换导致的(行变换不会交换未知数的位置)

关于完全主元素法, 有如下特点:
\begin{enumerate}
    \item 在选择主元时需要花费较多机器时间;
    \item 需要实时记录$x$的顺序变化;
\end{enumerate}

\subsubsection{列主元消去法}

在选择主元时仅考虑按列选取, 然后按行使之变到主元位置上, 再进行消元计算.

为了简单起见, 将通过一个例子实际演示用列主元Gauss消去法求解过程.

\begin{example}
    使用列主元消去法求解方程组
    \begin{equation*}
        \begin{cases}
            x_1+x_2+x_3=6\\
            2x_1+2x_2-x_3=3\\
            3x_1+x_3=6
        \end{cases}
    \end{equation*}
\end{example}

\begin{solution}
    线性方程组增广矩阵为
    \begin{equation*}
        \mqty(
            1&1&1&6\\
            2&2&-1&3\\
            3&0&1&6
        )
    \end{equation*}

    对第一列而言, 主元为3, 故将其作为第一行, 作行变换, 得
    \begin{equation*}
        \mqty(
            3&0&1&6\\
            2&2&-1&3\\
            1&1&1&6
        )\to\mqty(
            3&0&1&6\\
            0&2&-5/3&-1\\
            0&1&2/3&4
        )
    \end{equation*}

    对第二列而言, 2为主元, 故保持不变, 得
    \begin{equation*}
        \mqty(
            3&0&1&6\\
            0&2&-5/3&-1\\
            0&1&2/3&4
        )\to\mqty(
            3&0&1&6\\
            0&2&-5/3&-1\\
            0&0&9/6&9/2
        )
    \end{equation*}
    利用回代法, 得方程组解为$\transpose{(1,2,3)}$
\end{solution}

上述解答用程序语言可以实现如下:

\begin{lstlisting}
# 使用列主元Gauss消去法求解方程组 Exercise5-1.py

import numpy as np

#使用列主元Gauss消去法求解方程组
def gauss_elimination(A, b):
    n = A.shape[0]
    # 消元
    for j in range(n):
        # 选择最大值作为主元
        max_idx = j + np.argmax(np.abs(A[j:, j]))
        if j != max_idx:
            A[[j, max_idx]] = A[[max_idx, j]]
            b[[j, max_idx]] = b[[max_idx, j]]
        # 计算主元下面的元素
        pivot = A[j, j]
        for i in range(j+1, n):
            factor = A[i, j] / pivot
            A[i, :] -= factor * A[j, :]
            b[i] -= factor * b[j]
    # 回代
    x = np.zeros(n)
    for j in range(n-1, -1, -1):
        x[j] = (b[j] - np.dot(A[j, j+1:], x[j+1:])) / A[j, j]
    return x

# 计算方程组
A = np.array([[1,1,1],[2,2,-1],[3,0,1]], dtype=float)
b = np.array([6,3,6], dtype=float)
x = gauss_elimination(A, b)
print("x=", x)
\end{lstlisting}
程序计算结果与我们计算结果是一致的.

\subsubsection{Gauss-Jordan消去法}

基本思想: 同时消去对角线上方和下方的元素.

特点:
\begin{enumerate}
    \item 消元和回代同时进行;
    \item 乘除法次数比Gauss消去法大
\end{enumerate}

下面以一个线性方程组为例, 介绍如何实现Gauss-Jordan消去法:

\begin{example}
    使用Gauss-Jordan消去法求解方程组
    \begin{equation*}
        \begin{cases}
            x_1+x_2+x_3=6\\
            2x_1+x_2-x_3=1\\
            3x_1+x_3=6
        \end{cases}
    \end{equation*}
\end{example}

\begin{solution}
    对于Gauss-Jordan消去法, 我们不考虑主元. 增广矩阵为
    \begin{equation*}
        \mqty(
            1&1&1&6\\
            2&1&-1&1\\
            3&0&1&6
        )
    \end{equation*}
    首先进行第一次消元, 得到结果为
    \begin{equation*}
        \mqty(
            1&1&1&6\\
            2&1&-1&1\\
            3&0&1&6
        )\to\mqty(
            1&1&1&6\\
            0&-1&-3&-11\\
            0&-3&-2&-12
        )
    \end{equation*}
    对第二行进行消元, 对于Gauss-Jordan消去法, 我们需要同时考虑对角线上下的元素. 因此, 消元后结果如下:
    \begin{equation*}
        \mqty(
            1&1&1&6\\
            0&-1&-3&-11\\
            0&-3&-2&-12
        )\to\mqty(
            1&0&-2&-5\\
            0&1&3&11\\
            0&0&7&21
        )
    \end{equation*}
    可以看到, 经过第二步消元后, 第二列除主元素外所有元素都为0. 同理, 第三步消元得到结果为
    \begin{equation*}
        \mqty(
            1&0&-2&-5\\
            0&1&3&11\\
            0&0&7&21
        )\to\mqty(
            1&0&0&1\\
            0&1&0&2\\
            0&0&1&3
        )
    \end{equation*}
    因此解得方程组解为$\transpose{(1,2,3)}$
\end{solution}

\subsubsection{列主元Gauss-Jordan消去法}

与列主元Gauss消去法类似, 列主元Gauss-Jordan消去法步骤为:

第一步, 选择第一列的主元, 并将其换至第一行, 将第一个方程的系数变为1, 同时从其余$n-1$个方程中消去$x_1$;

第二步, 再第二列后$n-1$个元素中选择主元, 将第二个方程$x_2$系数变为1, 并从其他$n-1$个方程中消去$x_2$

$\cdots$

第$k$步, 在第$k$列后$n-k$个元素中选主元并换行, 将第$k$个方程的$x_k$系数变为1, 并从其他$n-1$个方程中消去变量$x_k$

消元公式为
\begin{equation*}
    \begin{cases}
        a_{kj}^{(k)}=\frac{a_{kj}^{(k-1)}}{a_{kk}^{(k-1)}}, j=k,k+1, \cdots,n+1\\
        a_{ij}^{(k)}=a_{ij}^{(k-1)}-a_{ik}^{(k-1)}a_{kj}^{(k)}\\
        j=k,k+1,\cdots,n+1\\
        i=1,2,\cdots,k-1,k+1,\cdots,n
    \end{cases}
\end{equation*}

对$k=1,2,\cdots$进行上述步骤至第$n$步后, 方程组可变为
\begin{equation*}
    \begin{cases}
        x_1=a_{1,n+1}^{(n)}\\
        x_2=a_{2,n+1}^{(n)}\\
        \vdots\\
        x_n=a_{n,n+1}^{(n)}
    \end{cases}
\end{equation*}
即为所求的解.

\subsubsection{Gauss-Jordan消去法应用}

对于线性方程组系, 如
\begin{equation*}
    AX=b_1,AX=b_2,\cdots,AX=b_m
\end{equation*}
其中,
\begin{equation*}
    A=\mqty(
        a_{11}&\cdots&a_{1n}\\
        \vdots&\ddots&\vdots\\
        a_{n1}&\cdots&a_{nn}
    ), X=\mqty(x_1\\\vdots\\x_n), b_i=\mqty(a_{1,n+1}^{(i)}\\a_{2,n+1}^{(i)}\\\vdots\\a_{n,n+1}^{(i)}), i=1,2,\cdots,n
\end{equation*}
因此上述方程组系可写作
\begin{equation*}
    AX=B=(b_1,\cdots,b_n)
\end{equation*}
其解为
\begin{equation*}
    X=\inv{A}B
\end{equation*}

特殊的, 设$A=(a_{ij})_{n\cross n}$是非奇异矩阵, $\abs*{A}\ne0$, 令
\begin{equation*}
    X=\inv{A}=(x_{ij})_{n\cross n}
\end{equation*}
由于$A\inv{A}=AX=I$, 因此求解$X$的过程等价于当$m=n, B=I$的方程组求解, 此时得到$X=\inv{A}$

\begin{example}
    用Gauss-Jordan消去法求解方程组$Ax=b$, 并求出$\inv{A}$, 其中,
    \begin{equation*}
        A=\mqty(
            1&3&1\\
            1&2&4\\
            5&1&2
        ), b=\mqty(10\\17\\13)
    \end{equation*}
\end{example}

\begin{solution}
    将系数矩阵, 单位矩阵和$b$组成增广矩阵, 并对其进行Gauss-Jordan消元过程, 可得过程为
    \begin{align*}
        &\mqty(
            1&3&1&1&&&10\\
            1&2&4&&1&&17\\
            5&1&2&&&1&13
        )\to\mqty(
            1&3&1&1&&&10\\
            0&-1&3&-1&1&&7\\
            0&-14&-3&-5&&1&-37
        )\\
        \to&\mqty(
            1&&10&-2&3&&31\\
            &1&-3&1&-1&0&-7\\
            &&-45&9&-14&1&-135
        )\to\mqty(
            1&&&0&-0.11&0.22&1\\
            &1&&0.4&-0.07&-0.07&2\\
            &&1&-0.2&0.31&-0.02&3
        )
    \end{align*}
    故方程组的解为$\transpose{(1,2,3)}$, 系数矩阵的逆为
    \begin{equation*}
        \inv{A}=\mqty(
            0&-0.11&0.22\\
            0.4&-0.07&-0.07\\
            -0.2&0.31&-0.02
        )
    \end{equation*}
\end{solution}

\section{解三对角方程组的追赶法}

求解线性方程组
\begin{equation*}
    \mqty(
        b_1&c_1&&&\\
        a_2&b_2&c_2&&\\
        &\ddots&\ddots&\ddots&\\
        &&a_{n-1}&b_{n-1}&c_{n-1}\\
        &&&a_n&b_n
    )\mqty(x_1\\x_2\\\vdots\\x_{n-1}\\x_n)=\mqty(d_1\\d_2\\\vdots\\d_{n-1}\\d_n)
\end{equation*}
且满足:
\begin{equation*}
    \begin{cases}
        \abs*{b_1}>\abs*{c_1}\\
        \abs*{b_i}\ge\abs*{a_i}+\abs*{c_i}, i=2,\cdots,n-1\\
        \abs*{b_n}>\abs*{a_n}
    \end{cases}
\end{equation*}

消元公式为
\begin{equation*}
    \begin{cases}
        r_1=\frac{c_1}{b_1}, y_1=\frac{d_1}{b_1}&\\
        r_k=\frac{c_k}{b_k-r_{k-1}a_k},&k=2,3,\cdots,n-1\\
        y_k=\frac{d_k-y_{k-1}a_k}{b_k-r_{k-1}a_k},&k=2,3,\cdots,n
    \end{cases}
\end{equation*}

回代公式
\begin{equation*}
    \begin{cases}
        x_n=y_n&\\
        x_k=y_k-r_kx_{k+1},&k=n-1,n-2,\cdots,1
    \end{cases}
\end{equation*}

\section{矩阵的三角分解法}

\subsection{Gauss消元法矩阵形式}

对于Gauss消元法, 每一步消元过程相当于左乘下三角矩阵$L_k$

记
\begin{equation*}
    A^{(2)}=L_1^{-1}A^{(1)}, b^{(2)}=L_1^{-1}b^{(1)}
\end{equation*}
其中, 
\begin{equation*}
    L_1^{-1}=\mqty(
        1&&&&\\
        -l_{21}&1&&&\\
        -l_{31}&0&1&&\\
        \vdots&\vdots&\vdots&\ddots&\\
        -l_{n1}&0&0&\cdots&1
    ), l_{i1}=\frac{a_{i1}^{(1)}}{a_{11}^{(1)}}, i=2,3,\cdots,n
\end{equation*}

同理, 有
\begin{equation*}
    A^{(3)}=L_2^{-1}A^{(2)}=L_2^{-1}L_1^{-1}A^{(1)}, b^{(3)}=L_2^{-1}L_1^{-1}b^{(1)}
\end{equation*}
其中,
\begin{equation*}
    L_2^{-1}=\mqty(
        1&&&&\\
        0&1&&&\\
        0&-l_{32}&1&&\\
        \vdots&\vdots&\vdots&\ddots&\\
        0&-l_{n2}&0&\cdots&1
    ), l_{i2}=\frac{a_{i2}^{(2)}}{a_{22}^{(2)}}, i=3,4,\cdots,n
\end{equation*}

一般地, 对$i=1,2,\cdots,n-1$, 令
\begin{equation*}
    L_i^{-1}=\mqty(
        1&&&&&\\
        0&1&&&&\\
        \vdots&&1&&&\\
        0&\cdots&-l_{i+1,i}&1&&\\
        \vdots&&\vdots&\vdots&\ddots&\\
        0&&-l_{ni}&&&1
    ), L_i=\mqty(
        1&&&&&&\\
        0&1&&&&&\\
        \vdots&&\ddots&&&&\\
        \vdots&&&1&&&\\
        0&&&l_{i+1,i}&1&&\\
        \vdots&&&\vdots&&\ddots&\\
        0&&&l_{ni}&&&1
    )
\end{equation*}
\begin{equation*}
    l_{ki}=\frac{a_{ki}^{(i)}}{a_{ii}^{(i)}}, k=i+1,\cdots,n
\end{equation*}
则
\begin{align*}
    A^{(n)}&=L_{n-1}^{-1}L_{n-2}^{-1}\cdots L_1^{-1}A^{(1)}\\
    b^{(n)}=L_{n-1}^{-1}L_{n-2}^{-1}\cdots L_1^{-1}b^{(1)}
\end{align*}
于是有
\begin{equation*}
    A^{(1)}=L_1L_2\cdots L_{n-1}A^{(n)}=LA^{(n)}=LU
\end{equation*}
将上式分解方式, 称为矩阵$A$的\emph{LU分解}.

\subsection{矩阵三角分解的定义}
\begin{definition}
    设$A$是$n\cross n$实矩阵, 如果存在下三角矩阵$L$与上三角矩阵$U$, 使得$A=LU$, 则称为矩阵$A$的\emph{三角分解}, 若存在单位下三角矩阵$L$, 对角矩阵$D$以及单位上三角矩阵$R$, 使得$A=LDR$, 则称为矩阵$A$的\emph{LDR分解}.
\end{definition}

若$L$为单位下三角矩阵而$U$是一般上三角矩阵, 则称该分解为\emph{Doolittle分解}; 若$L$为一般下三角矩阵而$U$是单位上三角矩阵, 则称该分解为\emph{Crout分解}.

\subsection{矩阵三角分解的存在性}

\begin{theorem}
    设$A$为$n\cross n$实矩阵, 若求解$AX=b$用顺序Gauss消去法能够完成, 即$a_{kk}^{(k)}\ne 0, k=1,2,\cdots,n$, 则矩阵$A$可分解为单位下三角矩阵$L$与上三角矩阵$U$的乘积, 即
    \begin{equation*}
        A=LU
    \end{equation*}
    且这种分解唯一.
\end{theorem}

若$A$已经实现了LU分解, 则将方程组$Ax=b$转化为$(LU)x=b$, 从而计算得
\begin{align*}
    Ly&=b\\
    Ux&=y
\end{align*}

若直接从矩阵$A$的元素得到计算$L,U$的递推公式, 而不需要中间步骤, 则称这种分解方法为\emph{直接三角分解法}

\subsubsection{直接三角分解法}
\begin{example}
    使用三角分解法解方程组
    \begin{equation*}
        \begin{cases}
            x_1+x_2+x_3=6\\
            4x_2-x_3=5\\
            2x_1-2x_2+x_3=1
        \end{cases}
    \end{equation*}
\end{example}

\begin{solution}
    系数矩阵为
    \begin{equation*}
        A=\mqty(
            1&1&1\\
            0&4&-1\\
            2&-2&1
        )
    \end{equation*}
    计算消元因子, 有
    \begin{align*}
        m_{21}&=0/1=0\\
        m_{31}&=2/1=2
    \end{align*}
    得行变换后矩阵为
    \begin{equation*}
        A\to\mqty(
            1&1&1\\
            0&4&-1\\
            0&-4&-1
        )
    \end{equation*}
    类似地, 消元因子
    \begin{equation*}
        m_{32}=-4/4=-1
    \end{equation*}
    矩阵变换
    \begin{equation*}
        \mqty(
            1&1&1\\
            0&4&-1\\
            0&0&-2
        )
    \end{equation*}

    因此, 由Gauss消去法得矩阵分解
    \begin{equation*}
        A=\mqty(
            1&0&0\\
            0&1&0\\
            2&-1&1
        )\mqty(
            1&1&1\\
            0&4&-1\\
            0&0&-2
        )
    \end{equation*}
    解方程组
    \begin{align*}
        \mqty(
            1&0&0\\
            0&1&0\\
            2&-1&1
        )\mqty(y_1\\y_2\\y_3)&=\mqty(6\\5\\1)\\
        \mqty(
            1&1&1\\
            0&4&-1\\
            0&0&-2 
        )\mqty(x_1\\x_2\\x_3)&=\mqty(y_1\\y_2\\y_3)
    \end{align*}
    解得
    \begin{equation*}
        \mqty(x_1\\x_2\\x_3)=\mqty(1\\2\\3)
    \end{equation*}
\end{solution}

\subsubsection{Doolittle三角分解法}
对于矩阵三角分解, 也可以直接使用\emph{待定系数法}, 称为Doolittle分解法.

\begin{example}
    利用Doolittle三角分解法重解上题
\end{example}

\begin{solution}
    待定系数法, 设
    \begin{equation*}
        L=\mqty(
            1&0&0\\
            l_{21}&1&0\\
            l_{31}&l_{32}&1
        ), U=\mqty(
            u_{11}&u_{12}&u_{13}\\
            0&u_{22}&u_{23}\\
            0&0&u_{33}
        )
    \end{equation*}

    利用矩阵分解表达式, 有
    \begin{equation*}
        \mqty(
            1&1&1\\
            0&4&-1\\
            2&-2&1
        )=\mqty(
            1&0&0\\
            l_{21}&1&0\\
            l_{31}&l_{32}&1
        )\mqty(
            u_{11}&u_{12}&u_{13}\\
            0&u_{22}&u_{23}\\
            0&0&u_{33}
        )
    \end{equation*}
    得
    \begin{align*}
        u_{11}&=1, u_{12}=1, u_{13}=1\\
        l_{21}&=0, u_{22}=4, u_{23}=-1\\
        l_{31}&=2, l_{32}=-1, u_{33}=-2
    \end{align*}
    故将矩阵分解为
    \begin{equation*}
        A=\mqty(
            1&0&0\\
            0&1&0\\
            2&-1&1
        )\mqty(
            1&1&1\\
            0&4&-1\\
            0&0&-2
        )
    \end{equation*}
\end{solution}

\begin{example}
    用矩阵三角分解法求解方程组
    \begin{equation*}
        \mqty(
            1&2&3\\
            2&5&2\\
            3&1&5
        )\mqty(x_1\\x_2\\x_3)=\mqty(14\\18\\20)
    \end{equation*}
\end{example}
\begin{solution}
    待定系数法, 
    \begin{equation*}
        \mqty(
            1&2&3\\
            2&5&2\\
            3&1&5
        )=\mqty(
            1&0&0\\
            l_{21}&1&0\\
            l_{31}&l_{32}&1
        )\mqty(
            u_{11}&u_{12}&u_{13}\\
            0&u_{22}&u_{23}\\
            0&0&u_{33}
        )
    \end{equation*}
    解得
    \begin{align*}
        u_{11}&=1, u_{12}=2, u_{13}=3\\
        l_{21}&=2, u_{22}=1, u_{23}=-4\\
        l_{31}&=3, l_{32}=-5, u_{33}=-24
    \end{align*}
    即
    \begin{equation*}
        L=\mqty(
            1&0&0\\
            2&1&0\\
            3&-5&1
        ), U=\mqty(
            1&2&3\\
            0&1&-4\\
            0&0&-24
        )
    \end{equation*}

    解方程组
    \begin{align*}
        Ly&=b\\
        Ux&=y
    \end{align*}
    可得
    \begin{equation*}
        x=\transpose{(1,2,3)}
    \end{equation*}
\end{solution}

\subsubsection{紧凑格式的Doolittle三角分解法}

考虑增广矩阵$(A;b)$, 将消元因子作为矩阵变换过程中的元素, 从而在变换过程中一次性得到$L,U,y$, 将这种过程称为\emph{紧凑格式的Doolittle三角分解法}.

\begin{example}
    使用紧凑格式的Doolittle三角分解法重解上题
\end{example}

\begin{solution}
    考虑增广矩阵
    \begin{equation*}
        \mqty(
            1&2&3&14\\
            2&5&2&18\\
            3&1&5&20
        )
    \end{equation*}
    消元因子$m_{21}=2, m_{31}=3$, 变换得
    \begin{equation*}
        \mqty(
            1&2&3&14\\
            2&1&-4&-10\\
            3&-5&-4&-22
        )
    \end{equation*}
    类似地, 消元因子$m_{32}=-5$, 故得变换后矩阵为
    \begin{equation*}
        \mqty(
            1&2&3&14\\
            2&1&-4&-10\\
            3&-5&-24&-72
        )
    \end{equation*}
    因此直接解得
    \begin{equation*}
        L=\mqty(
            1&0&0\\
            2&1&0\\
            3&-5&1
        ), U=\mqty(
            1&2&3\\
            0&1&-4\\
            0&0&-24
        ), y=\mqty(14\\-10\\-72)
    \end{equation*}

    解方程组$Ux=y$, 可得
    \begin{equation*}
        x=\transpose{(1,2,3)}
    \end{equation*}
\end{solution}

上述例题可使用下述程序代码求解:
\begin{lstlisting}
# 使用紧凑格式的Doolittle三角分解法求解线性方程组 Exercise5-2.py

import numpy as np

def doolittle_triangular_decomposition(A):
    """
    Doolittle三角分解法
    """
    n = A.shape[0]
    L = np.zeros((n, n))
    U = np.zeros((n, n)) 
    for i in range(n):
        L[i, i] = 1
        for j in range(i, n):
            U[i, j] = A[i, j] - sum(L[i, k] * U[k, j] for k in range(i))
            if i != j:
                L[j, i] = (A[j, i] - sum(L[j, k] * U[k, i] for k in range(i))) / U[i, i]
    return L, U

def forward_substitution(L, b):
    """
    前向替代
    """
    n = L.shape[0]
    x = np.zeros(n) 
    for i in range(n):
        x[i] = b[i] / L[i, i]
        for j in range(i):
            x[i] -= x[j] * L[i, j] / L[i, i]
    return x

def backward_substitution(U, b):
    """
    回代
    """
    n = U.shape[0]
    x = np.zeros(n)
    for i in reversed(range(n)):
        x[i] = (b[i] - sum(U[i, j] * x[j] for j in range(i+1, n))) / U[i, i]
    return x

def doolittle_solver(A, b):
    """
    使用Doolittle三角分解法求解线性方程组
    """
    L, U = doolittle_triangular_decomposition(A)
    y = forward_substitution(L, b)
    x = backward_substitution(U, y)
    return x

A = np.array([[1,2,3],[2,5,2],[3,1,5]], dtype=float)
b = np.array([14,18,20], dtype=float)
x = doolittle_solver(A, b)
print("x=", x)
\end{lstlisting}

类似地, 还有紧凑格式的列主元Doolittle三角分解法, 不做详细说明. 需要特别注意的是, 在变换矩阵过程中所得到的消元因子, \emph{不参与行变换的计算, 但需要参与行交换}. 

\section{Gauss消去法变形}

\subsection{矩阵LDR分解}

\begin{theorem}
    若$n$阶矩阵$A$的所有顺序主子式均不为0, 则矩阵$A$存在唯一分解式$A=LDR$, 其中$L, R$分别为$n$阶单位下三角矩阵和单位上三角矩阵, $D$为对角元素不为0的$n$阶对角矩阵. 上述分解方法称为$A$的LDR分解.
\end{theorem}

\begin{corollary}
    设$A$为$n$阶对称矩阵, 且$A$的所有顺序主子式均不为0, 则$A$可唯一分解为
    \begin{equation*}
        A+LD\transpose{L}
    \end{equation*}
    其中, $L$为单位下三角矩阵, $D$为对角矩阵
\end{corollary}

特殊地, 考虑对称正定矩阵, 其定义为
\begin{definition}
    设$A\in R^{n\cross n}$, 若$A$满足
    \begin{enumerate}
        \item \begin{equation*}
            \transpose{A}=A
        \end{equation*}

        \item $\forall x\in R^n$, 且$x\ne 0$, 有
        \begin{equation*}
            \transpose{x}Ax>0
        \end{equation*}
    \end{enumerate}
    则称$A$为\emph{对称正定矩阵}.
\end{definition}

关于对称正定矩阵, 有如下性质
\begin{enumerate}
    \item $A$是非奇异矩阵, 且$\inv{A}$是对称正定矩阵;
    \item $A$地顺序主子式均大于0, 即
    \begin{equation*}
        \det(A_k)>0, k=1,2,\cdots,n
    \end{equation*}
    \item $A$的特征值
    \begin{equation*}
        \lambda_i(A)>0, i=1,2,\cdots,n
    \end{equation*}
\end{enumerate}

对于对称正定矩阵, 矩阵有特殊的分解, 即\emph{平方根法}

\subsection{平方根法}

\begin{theorem}
    若$A$为对称正定矩阵, 则存在一个实的非奇异下三角矩阵, 使得$A=L\transpose{L}$, 且当限定的对角元素为正时, 这种分解是唯一的, 称为矩阵$A$的Cholesky分解.
\end{theorem}

对于平方根法, 其数值是稳定的, 即在计算过程中不需要选择主元, 但在计算过程中会引入平方根计算, 因此计算难度会较大(但计算量较一般的LU分解小)

\subsection{改进平方根法}

为避免开方运算, 我们改进平方根法, 即考虑将矩阵$A$分解为$LD\transpose{L}$, 其中$L$为单位下三角矩阵, $D$为对角矩阵.

通过分解, 将方程组转换为$LD\transpose{L}x=b$, 求解下列方程组
\begin{align*}
    Ly&=b
    \transpose{L}x=\inv{D}y
\end{align*}

下面将通过一个例子实际演示使用改进的平方根法求解方程组
\begin{example}
    用改进平方根法求解方程组
    \begin{equation*}
        \mqty(
            1&2&1\\
            2&3&1\\
            1&1&2
        )\mqty(x_1\\x_2\\x_3)=\mqty(4\\6\\4)
    \end{equation*}
\end{example}

\begin{solution}
    系数矩阵为对称矩阵, 则可以将其分解为
    \begin{equation*}
        \mqty(
            1&2&1\\
            2&3&1\\
            1&1&2
        )=\mqty(
            1&0&0\\
            l_{21}&1&0\\
            l_{31}&l_{32}&1
        )\mqty(
            d_1&&\\
            &d_2&\\
            &&d_3
        )\mqty(
            1&l_{21}&l_{31}\\
            0&1&l_{32}\\
            0&0&1
        )
    \end{equation*}

    解得
    \begin{equation*}
        L=\mqty(
            1&0&0\\
            2&1&0\\
            1&1&1
        ), D=\mqty(
            1&&\\
            &-1&\\
            &&2
        )
    \end{equation*}

    解方程
    \begin{equation*}
        Ly=b
    \end{equation*}
    得
    \begin{equation*}
        y=\transpose{(4,-2,2)}
    \end{equation*}
    
    解方程
    \begin{equation*}
        \transpose{L}x=\inv{D}y
    \end{equation*}
    其中
    \begin{equation*}
        \inv{D}=\mqty(
            1&&\\
            &-1&\\
            &&\frac{1}{2}
        )
    \end{equation*}
    得
    \begin{equation*}
        x=\transpose{(1,1,1)}
    \end{equation*}
\end{solution}

使用改进平方根法有如下优点:
\begin{enumerate}
    \item 计算量小, 是目前计算对称正定矩阵方程组得有效方法;
    \item 计算简单, 没有开方计算;
    \item 精度较高;
    \item 可以推广至非正定方程组的求解
\end{enumerate}

\section{线性方程组的性态和解的误差估计}

在求解线性方程$Ax=b$时, 系数$A$和$b$的误差对解$x$有影响. 

当$A$精确时, 设$b$有误差$\delta b$, 设得到的解为$x+\delta x$, 即
\begin{equation*}
    A(x+\delta x)=b+\delta b
\end{equation*}
有
\begin{equation*}
    \delta x=\inv{A}\delta b
\end{equation*}
由范数的性质可知,
\begin{equation*}
    \norm*{\delta x}\le\norm*{\inv{A}}\norm*{\delta b}
\end{equation*}
又因为
\begin{equation*}
    \norm*{b}=\norm*{Ax}\le\norm*{A}\norm*{x}\Rightarrow\frac{1}{\norm*{x}}\le\frac{\norm*{A}}{\norm*{b}}
\end{equation*}
即
\begin{equation*}
    \frac{\norm*{\delta x}}{\norm*{x}}\le\norm*{A}\norm*{\inv{A}}\frac{\norm*{\delta b}}{\norm*{b}}
\end{equation*}

同理, 设$b$精确, $A$有误差$\delta A$, 解为$x+\delta x$, 即
\begin{equation*}
    (A+\delta A)(x+\delta x)=b
\end{equation*}

可得
\begin{equation*}
    \frac{\norm*{\delta x}}{\norm*{x}}\le\frac{\norm*{A}\norm*{\inv{A}}\frac{\norm*{\delta A}}{\norm*{A}}}{1-\norm*{A}\norm*{\inv{A}}\frac{\norm*{\delta A}}{\norm*{A}}}
\end{equation*}

可以发现, $\norm*{A}\norm*{\inv{A}}$是衡量误差放大的重要因素, 称为\emph{条件数}, 记作$\cond{A}$.

\begin{definition}
    设线性方程组的系数矩阵是非奇异的, 若$\cond{A}$越大, 则称这个方程组\emph{越病态}, 反之, 若$\cond{A}$越小, 则称方程组\emph{越良态}.
\end{definition}

通常来说, 判断一个矩阵是否病态, 可采用如下方法:
\begin{enumerate}
    \item 行列式很大或很小;
    \item 元素之间相差较大数量级;
    \item 主元消去过程中出现小主元;
    \item 特征值相差较大数量级
\end{enumerate}