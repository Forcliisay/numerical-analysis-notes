\chapter{数值积分}

\section{引言}

\subsection{数值积分的必要性}
本章主要讨论如下形式的一元函数积分
\begin{equation*}
    I(f) = \int_{a}^{b}f(x)\dd{x}
\end{equation*}
在微积分里, 按Newton-Leibniz公式求定积分
\begin{equation*}
    I(f) = \int_{a}^{b}f(x)\dd{x} = F(b)-F(a)
\end{equation*}

要求$f(x)$的原函数$F(x)$:
\begin{itemize}
    \item 有解析表达式;
    \item 为初等函数.
\end{itemize}

实际上,
\begin{enumerate}
    \item $f(x)$的原函数$F(x)$不能用初等函数表示,如$\sin{x^2},\cos{x^2},\frac{\sin{x}}{x},\frac{1}{\ln{x}}$等;
    \item 被积函数的原函数可以用初等函数表示,但其表达式相当复杂,计算极不方便,如$x^2\sqrt{2x^2+3}$;
    \item $f(x)$没有表达式,只有数表形式。
\end{enumerate}
这时就需要积分的数值方法来帮忙了。

\subsection{数值积分的基本思想}
\subsubsection{数值积分的理论依据}
依据积分中值定理,对于连续函数$f(x)$,在$[a,b]$内存在一点$\xi $,使得
    \begin{equation*}
       I(f) = \int_{a}^{b}f(x)\dd{x} = (b-a)f(\xi )
    \end{equation*}
称$f(\xi )$为$f(x)$在区间$[a,b]$上的平均高度。

\subsubsection{求积公式的构造}

 若简单选取区间端点或中点的函数值作为平均高度,则求积公式(左矩形公式, 中矩形公式, 右矩形公式)如下:
    \begin{equation*}
        I(f)\approx f(a)(b-a)
    \end{equation*}
    \begin{equation*}
        I(f)\approx f(\frac{a+b}{2})(b-a)
    \end{equation*}
    \begin{equation*}
        I(f)\approx f(b)(b-a)
    \end{equation*}

\begin{definition}[两点求积公式]
    若取$a,b$两点,并令$f(\xi ) = \frac{f(a)+f(b)}{2}$,则可得到梯形公式(两点求积公式)
    \begin{equation*}
        I(f)\approx \frac{f(a)+f(b)}{2}(b-a)
    \end{equation*}
\end{definition}

\begin{definition}[三点求积公式/Simpson公式]
    若取三点,$a,b,c = \frac{a+b}{2}$并令$f(\xi ) = \frac{[f(a)+4f(c)+f(b)]}{6}$,则可得Simpson公式(三点求积公式)
    \begin{equation*}
        I(f)\approx (b-a)\frac{[f(a)+4f(c)+f(b)]}{6}
    \end{equation*}  
\end{definition}

\begin{definition}[机械求积]
    一般地,取区间$[a,b]$内$n+1$个点${x_i},(i=0,1,2,\cdots,n)$处的高度${f(x_i)},(i=0,1,2,\cdots,n)$,通过加权平均的方法近似地得出平均高度$f(\xi )$。这类求积方法称为机械求积。
    \begin{equation*}
        \int_{a}^{b}f(x)\dd{x} \approx (b-a)\sum_{i=0}^{n}\lambda_if(x_i)
    \end{equation*}
    或写成:
    \begin{equation*}
        \int_{a}^{b}f(x)\dd{x} \approx \sum_{k=0}^{n}A_kf(x_k)
    \end{equation*}
    其中$A_k$称为求积系数,$x_k$称为求积节点。
\end{definition}

记:

数值求积公式:
\begin{equation*}
     i_n(f) = \sum_{k=0}^{n}A_kf(x_k)   
\end{equation*}

求积公式余项(误差):
\begin{equation*}
    R(f) = I(f)-I_n(f) = \int_{a}^{b}f(x)\dd{x}-\sum_{k=0}^{n} A_kf(x_k)
\end{equation*}

构造或确定一个求积公式,要解决的问题包括:
\begin{enumerate}
    \item 确定求积系数$A_k$和求积节点$x_k$;
    \item 确定衡量求积公式好坏的标准;
    \item 求积公式的误差估计和收敛性分析.
\end{enumerate}

\subsection{求积公式的代数精度}

\begin{definition}
    称求积公式$I_n(f)=\sum_{k=0}^{n}A_kf(x_k)$具有$m$次代数精度,如果它满足如下两个条件:
    \begin{enumerate}
        \item 对所有$m$次多项式$P_m(x)$,有
        \begin{equation*}
            R(P_m) = I(P_m)-I_n(P_m) = 0
        \end{equation*}
        \item 存在$m+1$次多项式$P_{m+1}(x)$,使得
        \begin{equation*}
            R(P_{m+1}) = I(P_{m+1})-I_n(P_{m+1}) \ne 0
        \end{equation*}
    \end{enumerate}
    上述定义中的条件等价于:
    \begin{align*}
         R(x^k) &= I(x^k)-I_n(x^k) = 0,(0\leq k\leq m)\\
         R(x_{m+1})&\neq 0
    \end{align*}
    \begin{notice}
        梯形公式与中矩形公式都只具有1次代数精度。
    \end{notice}
\end{definition}



一般的,若要使求积公式(1)具有m次代数精度, 则只要使求积公式对$f(x) = 1,x,x^2,\cdots,x^m$都准确成立,即
\begin{equation*}
    \begin{cases}
        \sum_{k=0}^{n}A_k = b-a\\
        \sum_{k=0}^{n}A_kx_k = \frac{1}{2}(b^2-a^2)\\
        \sum_{k=0}^{n}A_kx^m_k = \frac{1}{m+1}(b^{m+1}-a^{m+1})
    \end{cases}
\end{equation*}

\section{插值型求积公式}

\subsection{定义}

\begin{definition}[拉格朗日插值公式]
    在积分区间$[a,b]$上,取$n+1$个节点$x_i,i=0,1,2,\cdots,n$作$f(x)$的$n$次代数插值多项式(拉格朗日插值公式):
    \begin{equation*}
        L_n(x) = \sum_{k=0}^{n}l_k(x)f(x_k)
    \end{equation*}
    则有$f(x) = L_n(x)+R_n(x)$
    其中,$R_n(x) = \frac{f^{(n+1)}(\xi)}{(n+1)!}w_{n+1}(x)$为插值余项。
    $w_{n+1}(x) = \prod_{j=0}^{n}(x-x_j)$
\end{definition}

于是有:
\begin{align*}
    \int_{a}^{b}f(x)\dd{x} &= \int_{a}^{b}L_n(x)\dd{x}+\int_{a}^{b}R_n(x)\dd{x} \\
    &= \sum_{j=0}^{n}[\int_{a}^{b}l_j(x)\dd{x}]f(x_j)+\int_{a}^{b}R_n(x)\dd{x}
\end{align*}
取
\begin{equation*}
    \int_{a}^{b}f(x)\dd{x} \approx \sum_{j=0}^{n}f(x_j)\int_{a}^{b}l_i(x)\dd{x}
\end{equation*}
其中$\int_{a}^{b} l_j(x)\dd{x}$为$A_j$。

$A_j = \int_{a}^{b} \prod_{i=0,i\neq j}^{n}\frac{(x-x_i)}{(x_j-x_i)}\dd{x}$由节点决定,与$f(x)$无关。


\subsection{截断误差与代数精度}

\subsubsection{截断误差}
\begin{align*}
    R(f) &= \int_{a}^{b}f(x)\dd{x} - \sum_{k=0}^{n}A_kf(x_k) \\
    &= \int_{a}^{b}[f(x)-L_n(x)]\dd{x} \\
    &= \int_{a}^{b}\frac{f^{n+1}(\xi_x)}{(n+1)!}\prod_{k=0}^{n}(x-x_k)\dd{x}
\end{align*}

\subsubsection{代数精度}
\begin{theorem}
    形如$\sum_{k=0}^{n}A_kf(x)$的求积公式至少有$n$次代数精度$\Leftrightarrow $该公式为插值型(即:$A_k = \int_{a}^{b}l_k(x)dx$)
    \begin{corollary}
        求积系数$A_k$满足:
        \begin{equation*}
            \sum_{k=0}^{n}A_k = b-a
        \end{equation*}
    \end{corollary}
\end{theorem}

\section{Newton-Cotes公式}

\subsection{Cotes系数}
取节点为等距分布:$x_i=a+ih, h=\frac{b-a}{n}, i=0,1,\cdots,n$
由此构造的插值型求积公式称为Newton-Cotes公式,此时
求积系数:
\begin{align*}%令$x=a+th$
    A_i &= \int_{x_0}^{x_n}\prod_{j\neq i}\frac{(x-x_j)}{x_i-x_j}\dd{x} \\
    &=\int_{0}^{n}\prod_{i\neq j}\frac{(t-j)h}{(i-j)h}\times h\dd{t} \\
    &= \frac{(b-a)(-1)^{n-i}}{ni!(n-i)!}\int_{0}^{n}\prod_{j\neq i}(t-j)\dd{t}\\
    &= C_i^{(n)}(b-a)
\end{align*}
其中,$C_i^{(n)}$为Cotes系数。

\subsection{Newton-Cotes公式}

\subsubsection{定义}

\begin{definition}[n阶闭型Newton-Cotes求积公式]
    记:
    \begin{equation*}
        C_i^{(n)} = \frac{(-1)^{n-i}}{i!(n-i)!n}\int_{0}^{n}[\prod_{k=0, k\neq i}^{n}(t-k)]\dd{t}
    \end{equation*}
    则$A_i = (b-a)C_i^{(n)},i=0,1,2,\cdots,n$
    求积公式变为
    \begin{equation*}
        \int_{a}^{b}f(x)\dd{x} \approx (b-a)\sum_{i=0}^{n}C_i^{(n)}f(x_i)
    \end{equation*}
    称上式为n阶闭型Newton-Cotes求积公式。
\end{definition}

\begin{notice}
    由式$C_i^{(n)} = \frac{(-1)^{n-i}}{i!(n-i)!n}\int_{0}^{n}[\prod_{k=0,k\neq i}^{n}(t-k)]\dd{t}$确定的Cotes系数只与$i$和$n$有关,与$f(x)$和积分区间$[a,b]$无关,且满足:
    (1)$C_i^{(n)} = C_{n-i}^{(n)}$
    (2)$\sum_{k=0}^{n}C_i^{(n)} = 1$
\end{notice}

\subsubsection{截断误差}
Newton-Cotes公式的截断误差为:
\begin{align*}
    R(f) &= \int_{a}^{b}\frac{f^{(n+1)}(\xi)}{(n+1)!}w_{(n+1)}(x)\dd{x} \\
    &= \frac{h^{n+2}}{(n+1)!}\int_{0}^{n}f^{(n+1)}(\xi)[\prod_{j=0}^{n}(t-j)]\dd{t},\xi \in (a,b)
\end{align*}

\subsubsection{代数精度}
\begin{theorem}
    当阶数$n$为偶数时,Newton-Cotes公式至少具有$n+1$次代数精度。
\end{theorem}

\begin{proof}
    只需验证当$n$为偶数时,Newton-Cotes公式对$f(x) = x^{n+1}$的余项为零。
    由于$f(x) = x^{n+1}$,所以$f^{(n+1)}(x) = (n+1)!$ 
    即得$R(f) = h^{n+2}\int_{0}^{n}\prod_{j=0}^{n}(t-j)\dd{t}$
    引进变换$t = u+\frac{n}{2}$,因为$n$为偶数,故$\frac{n}{2}$为整数,
    于是有
    \begin{equation*}
        R(f) = h^{n+2}\int_{-\frac{n}{2}}^{\frac{n}{2}}\prod_{j=0}^{n}(u+\frac{n}{2}-j)\dd{u}
    \end{equation*}
    据此可得$R(f)=0$,因为上述被积函数是个奇函数。
\end{proof}

\subsubsection{数值稳定性}

现在讨论舍入误差对计算结果产生的影响。
设用公式
\begin{equation*}
    I_n(f) = (b-a)\sum_{j=0}^{n}C_j^{(n)}f(x_j)
\end{equation*}
近似计算积分
\begin{equation*}
    I(f) = \int_{a}^{b}f(x)dx
\end{equation*}
时,其中函数值$f(x_j)$有误差$\varepsilon_j(j = 0,1,2,\cdots,n)$,而计算$C_j^{(n)}$没有误差,中间计算过程中的舍入误差也不考虑,
则在$I_n(f)$的计算中,由$\varepsilon_j$引起的误差为:
\begin{align*}
    e_n &= (b-a)\sum_{j=0}^{n}C_j^{(n)}f(x_i)-(b-a)\sum_{j=0}^{n}C_j^{(n)}(f(x_j)+\varepsilon_j) \\
    &=-(b-a)\sum_{j=0}^{n}C_j^{(n)}\varepsilon_j
\end{align*}
如果$C_j^{(n)}$都是正数,并设$\varepsilon = \max_{0\leq j\leq n}\abs{\varepsilon_j}$
则有
\begin{equation*}
    \abs{e_n}\leq \varepsilon(b-a)\sum_{j=0}^{n}\abs{C_j^{(n)}} = \varepsilon(b-a)
\end{equation*}
故$e_n$是有界的,即由$\varepsilon_j$引起的受到控制,不超过$\varepsilon$的$(b-a)$倍,保证了数值计算的稳定性。而当$n>7$时,$C_j^{(n)}$将出现负数,$\sum_{j=0}^{n}|C_j^{(n)}|$将随$n$增大,因而不能保证数值稳定性。
故高阶公式不宜采用,有实用价值的仅仅是几种低阶的求积公式。

\subsection{几种常见的低阶求积公式}
 
梯形公式:(代数精度=1)
\begin{align*}
    n&=1:\\
    C_0^{(1)} &= \frac{1}{2},C_1^{(1)} = \frac{1}{2}\\
    \int_{a}^{b}f(x)\dd{x}&\approx \frac{b-a}{2}[f(a)+f(b)]\\
    R[f] &= \int_{a}^{b}\frac{f''(\xi_x )}{2!}(x-a)(x-b)\dd{x} \\
    &= -\frac{(b-a)^3}{12}f''(\xi ),\xi \in [a,b]
\end{align*}

Simpson公式:(代数精度=3)
\begin{align*}
    n&=2:\\
    C_0^{(2)} &= \frac{1}{6},C_1^{(2)} = \frac{2}{3},C_2^{(2)} = \frac{1}{6}\\
    \int_{a}^{b}f(x)\dd{x}&\approx \frac{b-a}{6}[f(a)+4f(\frac{a+b}{2})+f(b)]\\
    R[f] &= -\frac{(b-a)^5}{2880}f^{(4)}(\xi ),\xi \in (a,b)
\end{align*}

Cotes公式:(代数精度=5 )
\begin{align*}
    n&=4:\\
    \int_{a}^{b}f(x)\dd{x}&\approx \frac{b-a}{90}[7f(x_0)+32f(x_1)+12f(x_2)+32f(x_3)+7f(x_4)]\\
    x_k &= a+kh,h = \frac{b-a}{4},k = 0,1,2,3,4\\
    R[f] &= -\frac{8}{945}(\frac{b-a}{4})^7f^{(6)}(\xi ) \\
    &= -\frac{2(b-a)}{945}(\frac{b-a}{4})^6f^{(6)}(\xi ),\xi \in [a,b]
\end{align*}

\subsection{复化求积公式}

\begin{definition}[复化梯形公式]
    $h = \frac{b-a}{h},x_k=a+kh(k=0,\cdots,n)$
    在每个$[x_k,x_k+1]$上用梯形公式:
    \begin{align*}
        \int_{x_k}^{x_{k+1}}f(x)\dd{x}&\approx \frac{x_{k+1}-x_k}{2}[f(x_k)+f(x_{k+1})],k=0,\cdots,n-1\\
        \int_{a}^{b}f(x)\dd{x}&\approx \sum_{k=0}^{n-1}\frac{h}{2}[f(x_k)+f(x_{k+1})] = \frac{h}{2}[f(a)+2\sum_{k=1}^{n-1}f(x_k)+f(b)] = T_n\\
        R[f] &= \sum_{k=0}^{n-1}[-\frac{h^3}{12}f''(\eta_k)] = -\frac{h^2}{12}(b-a)\frac{\sum_{k=0}^{n-1}f''(\eta_k)}{n} = -\frac{h^2}{12}(b-a)f''(\eta),\eta \in (a,b)
    \end{align*}
\end{definition}

\begin{definition}[复化Simpson公式]
    $h = \frac{b-a}{n},x_k = a+kh(k=0,\cdots,n)$
    在每个$[x_k,x_k+1]$上使用Simpson公式:
    \begin{align*}
        \int_{x_k}^{x_{k+1}}f(x)\dd{x}&\approx \frac{h}{6}[f(x_k)+4f(x_{k+\frac{1}{2}})+f(x_{k+1})]\\
        \int_{a}^{b}f(x)\dd{x}&\approx \sum_{k=0}^{n-1}\int_{x_k}^{x_{k+1}}f(x)\dd{x} \approx \frac{h}{6}[f(a)+4\sum_{k=0}^{n-1}f(x_{k+\frac{1}{2}})+2\sum_{k=1}^{n-1}f(x_k)+f(b)] = S_n\\
        R[f] &= -\frac{b-a}{2880}h^4f^{(4)}(\xi ) = -\frac{b-a}{180}(\frac{h}{2})^4f^{(4)}(\xi ),\xi \in (a,b)
    \end{align*}
\end{definition}

\begin{definition}[复化Cotes公式]
    $h = \frac{b-a}{n},x_k = a+kh(k=0,\cdots,n)$
    \begin{align*}
        \int_{x_k}^{x_{k+1}}f(x)\dd{x}&\approx \frac{h}{90}[7f(x_k)+32f(x_{k+\frac{1}{4}})+12f(x_{k+\frac{1}{2}})+32f(_{k+\frac{3}{4}})+7f(x_{k+1})]\\
        \int_{a}^{b}f(x)\dd{x}&\approx \sum_{k=0}^{n-1}[7f(x_k)+32f(x_{k+\frac{1}{4}})+12f(x_{k+\frac{1}{2}})+32f(_{k+\frac{3}{4}})+7f(x_{k+1})] = C_n\\
        R[f] &= -\frac{2(b-a)}{945}(\frac{h}{4})^6f^{(6)}(\xi ),\xi \in (a,b)
    \end{align*}
\end{definition}

\begin{example}
    利用数据表
\end{example}

\subsubsection{复化梯形公式的误差估计}

误差先验估计式:
给定精度$\varepsilon $,如何求$n$?
\begin{example}
     要求$|I-T_n|<\varepsilon$,如何判断$n=$?
    由
    \begin{equation*}
        R[f] = I(f)-T_n(f) = -\frac{h^2}{12}(b-a)f''(\xi )
    \end{equation*}
    若记$M_2 = \max_{a\leq x\leq b}|f''(x)|$
    则可令
    \begin{equation*}
    \abs{\frac{h^2}{12}(b-a)f''(\xi )}\leq \frac{h^2}{12}(b-a)M_2 \leq \varepsilon
    \end{equation*}
    其中,$h=\frac{b-a}{n}$,于是,$n\geq \sqrt{\frac{(b-a)^3M_2}{12\varepsilon}} $
    \begin{align*}
        R[f] &= -\frac{h^2}{12}(b-a)f''(\varepsilon) \\
        &= -\frac{h^2}{12}\sum_{k=1}^{n}[f''(\xi_k)\cdot h] \approx -\frac{h^2}{12}\int_{a}^{b}f''(x)dx \\
        &= -\frac{h^2}{12}[f'(b)-f'(a)]
    \end{align*}
\end{example}
上例中若要求$|I-T_n|<10^{-6}$,则
\begin{equation*}
    |R_n[f]| \approx \frac{h^2}{12}|f'(1)-f'(0)| = \frac{h^2}{6}<10^{-6}
    \Rightarrow h<0.00244949 
\end{equation*}
即取$n=409$
通常采取将区间不断对分的方法,即取$n = 2^k$
上例中$2^k \geq 409 \Rightarrow k=9$时,$T_512=3.141592502$

误差后验估计式:
注意到区间再次对分时
\begin{align*}
    R_2n[f]&\approx -\frac{1}{12}[f'(b)-f'(a)](\frac{h}{2})^2 \\
    &\approx \frac{1}{4}R_n[f] \\
    \Rightarrow \frac{I-T_{2n}}{I-T_n} &\approx \frac{1}{4} \\
    \Rightarrow I-T_{2n} &\approx \frac{1}{3}(T_{2n}-T_n) = \frac{1}{4-1}(T_{2n}-T_n) 
\end{align*}
可用来判断迭代是否停止。

\subsubsection{复化Simpson公式的误差估计}

误差先验估计式:
\begin{align*}
    R[f] &= I-S_n = -\frac{b-a}{180}(\frac{h}{4})^4f^{(4)}(\xi ) \\
    &\approx -\frac{1}{180}[f^{(3)}(b)-f^{(3)}(a)](\frac{h}{2})^4
\end{align*}

误差后验估计式:
\begin{equation*}
    I-S_{2n} \approx \frac{1}{15}(S_{2n}-S_n) = \frac{1}{4^2-1}(S_{2n}-S_n)
\end{equation*}

\subsubsection{复化Cotes公式的误差估计}

误差先验估计式:
\begin{align*}
    R[f] &= I-C_n = -\frac{2(b-a)}{945}(\frac{h}{4})^6f^{(6)}(\xi ),\xi \in (a,b) \\
    &\approx -\frac{2}{945}[f^{(5)}(b)-f^{(5)}(a)](\frac{h}{4})^6
\end{align*}

误差后验估计式:
\begin{equation*}
    I-C_{2n} \approx \frac{1}{63}(C_{2n}-C_n) = \frac{1}{4^3-1}(C_{2n}-C_n)
\end{equation*}

\subsubsection{复化求积公式的收敛速度(阶)}
\begin{definition}[$p$阶收敛]
    若一个复化求积公式的误差满足$\lim_{h\rightarrow 0}\frac{R[f]}{h^p} = C<\infty $且$C\neq 0$,则称该公式是$p$阶收敛的。
\end{definition}
\begin{notice}
    根据上述定义不难验证
    \begin{equation}
        T_n\sim O(h^2),S_n\sim O(h^4),C_n\sim O(h^6)
    \end{equation}
\end{notice}

\section{Romberg算法}

\subsection{Romberg求积公式}

\begin{example}
    计算
    \begin{equation*}
    \pi = \int_{0}^{1}\frac{4}{1+x^2}\dd{x}    
    \end{equation*}
    若取$\varepsilon = 10^{-6}$,则利用复化求积公式进行计算时,须将区间对分9次,得到$T_{512} = 3.141592502$
\end{example}

考察$\frac{I-T_{2n}}{I-T_n} \approx \frac{1}{4}$,若由$I \approx \frac{4T_{2n}-T_n}{4-1} = \frac{4}{3}T_{2n}-\frac{1}{3}T_n$来计算$I$效果是否好些?
\begin{equation*}
    \frac{4}{3}T_8-\frac{1}{3}T_4 = 3.141592502 = S_4
\end{equation*}
一般有:
\begin{itemize}
    \item $\frac{4T_{2n}}{4-1} = S_n$
    \item $\frac{4^2S_{2n}-S_n}{4^2-1} = C_n$
    \item $\frac{4^3C_{2n}-C_n}{4^3-1} = R_n$(Romberg求积公式)
\end{itemize}

\subsection{理查德森外推加速法}

利用低阶公式产生高精度的结果。
设对于某一$h \neq 0$,有公式$T_0(h)$近似计算某一未知值$I$。
由Taylor展开得到:
\begin{equation*}
    T_0(h)-I = \alpha_1h+\alpha_2h^2+\alpha_3h^3+\cdots
\end{equation*}
现将$h$对分,得:
\begin{equation*}
    T_0(\frac{h}{2})-I = \alpha_1(\frac{h}{2})+\alpha_2(\frac{h}{2})^2+\alpha_3(\frac{h}{2})^3+\cdots
\end{equation*}

如何将公式精度由$O(h)$提高到$O(h^2)$?
\begin{equation*}
    \frac{2T_0(\frac{h}{2})-T_0(h)}{2-1}-I = -\frac{1}{2}\alpha_2h^2-\frac{3}{4}\alpha_3h^3-\cdots
\end{equation*}
即:
\begin{align*}
    T_1(h) &= \frac{2T_0(\frac{h}{2})-T_0(h)}{2-1} = I+\beta_1h^2+\beta_2h^3+\cdots\\
    T_2(h) &= \frac{2^2T_1(\frac{h}{2})-T_1(h)}{2^2-1} = I+\gamma_1h^3+\gamma_2h^4+\cdots\\
    \Rightarrow T_m(h) &= \frac{2^mT_{m-1}(\frac{h}{2})-T_{m-1}(h)}{2^m-1} = I+\sigma_1h^{m+1}+\sigma_2h^{m+2}+\cdots
\end{align*}


\subsection{复化梯形公式的渐进展开式}
\begin{theorem}
    设$f(x)\in C^{\infty}[a,b]$,则成立
    \begin{equation*}
        T_n(f) = I+a_1h^2+a_2h^4+\cdots +a_kh^{2k}+\cdots 
    \end{equation*}
    其中,$h=\frac{b-a}{h}$,系数$a_k(k=1,2,\cdots)$与$h$无关。
\end{theorem}


加速公式:
\begin{equation*}
    T_m^{(k)} = \frac{4^mT_{m-1}^{(k+1)}-T_{m-1}^{(k)}}{4^m-1},k = 1,2,\cdots; m = 1,2,\cdots
\end{equation*}

\begin{example}
    求形如$\int_{-1}^{1}f(x)dx \approx A_0f(x_0)+A_1f(x_1)$的两点求积公式。
    (1)求梯形公式(即以$x_0 = -1,x_1 = 1$为节点的插值型求积公式)立即可得。
    \begin{equation*}
        \int_{-1}^{1}f(x)dx \approx f(-1)+f(1)
    \end{equation*}
    (只具有1次代数精度)
    (2)若对求积公式中的四个待定系数$A_0,A_1,x_0,x_1$适当选取,使求积公式对$f(x) = 1,x,x^2,x^3$都准确成立,则$A_0,A_1,x_0,x_1$需满足如下方程组:
    \begin{equation*}
        \begin{cases}
            A_0 + A_1 = b-a\\
            A_0x_0 + A_1x_1 = \frac{1}{2}(b^2-a^2)\\
            A_0x_0^2 + A_1x_1^2 = \frac{1}{3}(b^3-a^3)\\
            A_0x_0^3 + A_1x_1^3 = \frac{1}{4}(b^4-a^4)\\
        \end{cases}
    \end{equation*}
\end{example}


\section{Gauss型求积公式}

\begin{definition}[Gauss型求积公式]
    具有$2n+1$次代数精度的插值型求积公式
    \begin{equation*}
        \int_{a}^{b}\rho(x)f(x)\dd{x}\approx \sum_{k=0}^{n}A_kf(x_k)
    \end{equation*}
    称为Gauss型求积公式。
\end{definition}

\begin{notice}
    Gauss型求积公式是代数精度最高的插值型求积公式。
\end{notice}

事实上,对于插值型求积公式
\begin{equation*}
    \int_{a}^{b}\rho(x)f(x)\dd{x}\approx \sum_{k=0}{n}A_kf(x_k)
\end{equation*}
其代数精度最高可达$2n+1$次(Gauss型求积公式)。
考虑$2n+2$次多项式$w_{n+1}^2(x)$,其中$w_{n+1}(x) = \prod_{j=0}^{n}(x-x_j)$
\begin{equation*}
    \int_{a}^{b}\rho(x)w_{n+1}^2(x)\dd{x} > 0
\end{equation*}
而$\sum_{k=0}^{n}A_kw^2_{n+1}(x_k) = 0$
故
\begin{equation*}
    \int_{a}^{b}\rho(x)w_{n+1}^2(x)\dd{x} \neq \sum_{k=0}^{n}A_kw^2_{n+1}(x_k) = 0
\end{equation*}

\subsection{高斯型求积公式的构造}

1.待定系数法
将节点$x_0,\cdots,x_n$以及系数$A_0,\cdots,A_n$都作为待定系数。并令求积公式对$f(x) = 1,x,x_2,\cdots,x^{2n-1}$精确成立可得非线性方程组:
\begin{equation*}
    \begin{cases}
        \sum_{k=0}^{n}A_k = b-a\\
        \sum_{k=0}^{n}A_kx_k = \frac{1}{2(b^2-a^2)}\\
        \vdots \\
        \sum_{k=0}^{n}A_kx_k^{2n+1} = \frac{1}{2n+2}(b^{2n+2}-a^{2n+2})\\
    \end{cases}
\end{equation*}

求解该方程组即可得相应的求积节点与求积系数。
\begin{example}
    求$\int_{0}^{1}\sqrt{x}f(x)dx$的2点Gauss公式。    
\end{example}

\begin{solution}
    设$\int_{0}^{1}\sqrt{x}f(x)dx \approx A_0f(x_0)+A_1f(x_1)$,应有3次代数精度。令上述公式$f(x) = 1,x,x^2,x^3$精确成立可得
    \begin{equation*}
        \begin{cases}
            \frac{2}{3} = A_0 + A_1\\
            \frac{2}{5} = A_0x_0 + A_1x_1\\
            \frac{2}{7} = A_0x_0^2 + A_1x_1^2\\
            \frac{2}{9} = A_0x_0^3 + A_1x_1^3\\
        \end{cases}
    \end{equation*}
    
    (不是线性方程组,不易求解。)
    解得:
    \begin{equation*}
        \begin{cases}
            x_0 \approx 0.8212\\
            x_1 \approx 0.2899\\
            A_0 \approx 0.3891\\
            A_1 \approx 0.2776\\
        \end{cases}
    \end{equation*}
\end{solution}

2.正交多项式法
\begin{theorem}
    $x_0,\cdots,x_n$为Gauss点$\Leftrightarrow w(x) = \prod_{k=0}^{n}(x-x_k)$与任意次数不大于n的多项式$P(x)$(带权)正交。
\end{theorem}
\begin{proof}
    "$\Rightarrow$":
    $x_0,\cdots,x_n$为Gauss点,则公式
    \begin{equation*}
        \int_{a}^{b}\rho(x)f(x)\dd{x}\approx \sum_{k=0}^{n}A_kf(x_k)
    \end{equation*}
    具有$2n+1$次代数精度。

    对任意次数不大于$n$的多项式$P_n(x)$,$P_n(x)w(x)$的次数不大于$2n+1$,则代入公式应精确成立:
    \begin{equation*}
        \int_{a}^{b}\rho(x)P_m(x)w(x)\dd{x} = \sum_{k=0}^{n}A_kP_m(x_k)w(x_k) = 0
    \end{equation*}
    "$\Leftarrow$":
    要证明$x_0,\cdots,x_n$为Gauss点,即要证公式对任意次数不大于$2n+1$的多项式$P_{2n+1}(x)$精确成立,即证明:
    \begin{equation*}
        \int_{a}^{b}\rho(x)P_m(x)\dd{x} = \sum_{k=0}^{n}A_kP_m(x_k)
    \end{equation*}
    设$P_{2n+1}(x) = w(x)q(x)+r(x)$
    \begin{align*}
        \int_{a}^{b}\rho(x)P_m(x)\dd{x} &= \int_{a}^{b}\rho(x)w(x)q(x)\dd{x} + \int_{a}^{b}\rho(x)r(x)\dd{x} \\
        &= \sum_{k=0}^{n}A_kr(x_k) \\
        &= \sum_{k=0}^{n}A_kP_m(x_k)
    \end{align*}
\end{proof}

正交多项式族${\varphi_0,\varphi_1,\cdots,\varphi_n,\cdots}$有性质:任意次数不大于$n$的多项式$P(x)$必与$\varphi_{n+1}$正交。
$\Rightarrow$若取$w(x)$为其中的$\varphi_{n+1}$,则$\varphi_{n+1}$的根就是Gauss点。

\begin{example}
    使用正交多项式重解上例
\end{example}
\begin{solution}
    \begin{equation*}
        \int_{0}^{1}\sqrt{x}f(x)\dd{x} \approx A_0f(x_0)+A_1f(x_1)
    \end{equation*}

    Step1:构造正交多项式$\varphi_2$
    设$\varphi_0(x) = 1,\varphi_1(x) = x+a,\varphi_2(x) = x^2+bx+c$
    \begin{enumerate}
        \item $(\varphi_0,\varphi_1) = 0 \Rightarrow \int_{0}^{1}\sqrt{x}(x+a)\dd{x} = 0 \Rightarrow a=-\frac{3}{5}$
        \item $(\varphi_0,\varphi_2) = 0 \Rightarrow \int_{0}^{1}\sqrt{x}(x^2+bx+c)\dd{x} = 0 \Rightarrow b = -\frac{10}{9}$
        \item $(\varphi_1,\varphi_2) = 0 \Rightarrow \int_{0}^{1}\sqrt{x}(x-\frac{3}{5})(x+bx+c)\dd{x} = 0 \Rightarrow c = \frac{5}{21}$
    \end{enumerate}
    即$\varphi_2(x) = x^2-\frac{10}{9}x+\frac{5}{21}$

    Step2:求$\varphi_2 = 0$的2个根,即为Gauss点$x_o,x_1$
    \begin{equation*}
        x_{0;1} = \frac{\frac{10}{9}\pm \sqrt{(\frac{10}{9})^2-\frac{20}{21}}}{2}
    \end{equation*}

    Step3:代入$f(x) = 1,x$以求解$A_0,A_1$

    结果与前一方法相同:$x_0 \approx 0.8212,x_1 \approx 0.2899,A_0 \approx 0.3891,A_1 \approx 0.2776$
    利用此公式计算$\int_{0}^{1}\sqrt{x}e^xdx$的值
    \begin{align*}
        \int_{0}^{1}\sqrt{x}e^xdx &\approx A_0e^{x_0} + A_1e^{x_1} \\
        &= 0.3891 \times e^{0.8212} + 0.2776 \times e^{0.2899} \approx 1.2555
    \end{align*}
\end{solution}

\begin{notice}
    构造正交多项式也可以利用$L-S$拟合中介绍过的递推式进行。
\end{notice}

\subsection{几种常见的Gauss型求积公式}

\subsubsection{Gauss-Legendre求积公式}

Legendre多项式:
定义在$[-1,1]$,上$\rho(x)\equiv 1$
\begin{equation*}
    P_k(x) = \frac{1}{2^kk!}\dv[k]{x}(x^2-1)^k
\end{equation*}
满足:
\begin{equation*}
    (P_k,P_l) = 
    \begin{cases}
        0& k\neq l \\
        \frac{2}{2k+1}& k=l\\
    \end{cases}
\end{equation*}

$P_0 = 1,P_1 = x$,递推公式:$(k+1)P_{k+1} = (2k+1)xP_k-kP_{k-1}$

Gauss-Legendre求积公式:
\begin{equation*}
    \int_{-1}^{1}f(x)\dd{x} \approx \sum_{k=0}^{n}A_kf(x_k)
\end{equation*}
其中,求积节点为$P_{n+1}$的根(求积系数通过解线性方程组得到)。

区间$[a,b]$上的Gauss-Legendre公式:
\begin{align*}
    \int_{a}^{b}f(x)\dd{x} &= \int_{-1}^{1}f(\frac{b-a}{2}t+\frac{b+a}{2})\cdot \frac{b-a}{2}\dd{t}\\
    &\approx\sum_{k=0}^{n}A_k\frac{b-a}{2}f(\frac{b-a}{2}t_k+\frac{b+a}{2})
\end{align*}
其中,$t_k$为n+1次Legendre多项式$P_{n+1}$的根。

\subsubsection{Gauss-Chebyshev求积公式:}

Chebyshev多项式:
定义在$[-1,1]$上,$\rho(x) = \frac{1}{\sqrt{1-x^2}}$
\begin{equation*}
    T_k(x) = cos(k \times \arccos{x})
\end{equation*}
$T_{n+1}$的根为:
\begin{equation*}
    x_k = cos(\frac{2k+1}{2n+2}\pi ), k = 0,\cdots, n
\end{equation*}
以此为节点构造公式
\begin{equation*}
    \int_{-1}^{1}\frac{f(x)}{\sqrt{1-x^2}}\dd{x} \approx \sum_{k=0}^{n}A_kf(x_k) = \frac{\pi}{n+1}\sum_{k=0}^{n}f(x_l)
\end{equation*}
称为Gauss-Chebyshev多项式。
\begin{notice}
    到积分端点$\pm $可能是积分的奇点,用普通Newton-Cotes公式在端点会出问题。
    而Gauss公式可能避免此问题的发生。
\end{notice}

\subsubsection{第二类Gauss-Chebyshev求积公式}

第二类Chebyshev多项式:
区间$[-1,1]$上,带权$\rho(x) = \sqrt{1-x^2}$
\begin{equation*}
    U_n(x) = \frac{\sin[(n+1)\arccos{x}]}{\sqrt{1-x^2}}\    (n=0,1,2,\cdots)
\end{equation*}
以$U_{n+1}$的零点作为求积节点构造的公式
\begin{align*}
    \int_{-1}^{1}\sqrt{1-x^2}f(x)\dd{x} &\approx \sum_{k=0}^{n}A_kf(x_k) \\
    &= \frac{\pi}{n+1}\sum_{k=0}^{n}\sin^2\frac{k\pi}{n+1}f(\cos\frac{k\pi}{n+1})
\end{align*}
称为第二类Gauss-Chebyshev公式。

\subsubsection{Gauss-Laguerre求积公式}
Laguerre多项式:
\begin{align*}
    L_n(x) &= e^x\dv[n]{x}(x^ne^{-x}), x\in [0,+\infty], n=0,1,2,\cdots\\
    \rho(x) &= e^{-x}
\end{align*}
以$n+1$次Laguerre多项式的零点为求积节点构造的公式
\begin{equation*}
    \int_{0}^{+\infty}e^{-x}f(x)\dd{x} \approx \sum_{k=0}^{n}A_kf(x_k)
\end{equation*}
称为Gauss-Laguerre求积公式。

\subsubsection{Gauss-Hermite求积公式}
Hermite多项式:
\begin{align*}
    H_n(x) &= (-1)^n\cdot e^{x^2}\cdot\dv[n]{x}(e^{-x^2}),x\in(-\infty ,+\infty), n = 0,1,2,\cdots\\
    \rho(x) &= e^{-x^2}
\end{align*}
以$n+1$次Hermite多项式的零点为求积节点构造的公式
\begin{equation*}
    \int_{-\infty}^{+\infty}e^{-x^2}f(x)\dd{x} \approx \sum_{k=0}^{n}A_kf(x_k)
\end{equation*}
称为Gauss-Hermite求积公式。

\subsection{Gauss公式的余项}
\begin{align*}
    R[f] &= \int_{a}^{b}f(x)\dd{x} -\sum_{k=0}^{n}A_kf(x_k) \\
    &=\int_{a}^{b}f(x)\dd{x} -\sum_{k=0}^{n}A_kP(x_k) \\
    &= \int_{a}^{b}f(x)\dd{x} - \int_{a}^{b}P(x)\dd{x} \\
    &= \int_{a}^{b}[f(x)-P(x)]\dd{x}
\end{align*}

\subsection{Hermite多项式的余项}
Hermite多项式的插值条件为:$H(x_k) = f(x_k),H'(x_k) = f'(x_k)$
Hermite插值余项为
\begin{equation*}
    R(x) = f(x) - H(x) = \frac{f^{(2n+2)}(\xi_x)}{(2n+2)!}\omega^2(x)
\end{equation*}
其中,$\omega(x) = (x-x_0)\cdots (x-x_n)$,$\xi_x$与$x$有关。
\begin{align*}
    \Rightarrow R[f] &= \int_{a}^{b}R(x)\dd{x} \\
    &= \int_{a}^{b}\frac{f^{(2n+2)}(\xi_x)}{(2n+2)!}\omega^2(x)\dd{x} \\
    &= \frac{f^{(2n+2)}(\xi_x)}{(2n+2)!}\int_{a}^{b}\omega^2(x)\dd{x}
\end{align*}
其中, $\xi \in (a,b)$
\subsection{Gauss型求积公式的收敛性}

\begin{theorem}
    若$f(x) \in C[a,b]$,则Gauss型求积公式所求积分值序列${I_n(f) = \sum_{k=1}^{n}A_kf(x_k)}$收敛于积分值$I(f)$,即
    \begin{equation*}
        \lim_{n\to\infty}\sum_{k=1}^{n}A_kf(x_k) = \int_{a}^{b}\rho(x)f(x)\dd{x}
    \end{equation*}
\end{theorem}

\subsection{Gauss型求积公式的数值稳定性}

\begin{theorem}
    Gauss型求积公式的求积系数都大于零,从而Gauss型求积公式是数值稳定的。
\end{theorem}

\subsubsection{复化Gauss-Legender求积公式}

$h = \frac{b-a}{n},x_k = a+kh \   (k=0,\cdots ,n)$
\begin{align*}
    \int_{x_k}^{x_{k+1}}f(x)\dd{x} &= \frac{h}{2}\int_{-1}^{1}f(\frac{h}{2}t+a+(k+\frac{1}{2})h)\dd{t} \\
    &\approx \frac{h}{2}\sum_{k=0}^{n}A_kf(\frac{h}{2}t_k+a+(k+\frac{1}{2})h)
\end{align*}
其中,$t_k$为$n+1$次Legendre多项式$P_{n+1}$的根。从而
\begin{equation*}
    \int_{a}^{b}f(x)\dd{x} \approx \frac{b-a}{2n}\sum_{k=0}^{n-1}[\sum_{j=0}^{n}A_j^{(k)}f(\frac{h}{2}t_j^{(k)}+a+(k+\frac{1}{2})h)]
\end{equation*}

\subsection{复化两点Gauss-Legender求积公式}
\begin{equation*}
    \int_{a}^{b}f(x)\dd{x} \approx \frac{b-a}{2n}\sum_{k=0}^{n-1}[F_k(-\frac{1}{\sqrt{3}})+F_k(\frac{1}{\sqrt{3}})]
\end{equation*}
其中,$F_k(t) = f(\frac{h}{2}t+a+(k+\frac{1}{2})h)$

\subsection{复化三点Gauss-Legender求积公式}
\begin{equation*}
    \int_{a}^{b}f(x)\dd{x} \approx \frac{b-a}{2n}\sum_{k=0}^{n-1}[\frac{5}{9}F_k(-\sqrt{\frac{3}{5}})+\frac{8}{9}F_k(0)+\frac{5}{9}F_k(\sqrt{\frac{3}{5}})]
\end{equation*}


% \section{第四章小结}
% 重点:
% \begin{enumerate}
%     \item 代数精度的定义(数值求积公式的构造及代数精度的差别)
%     \item 插值型求积公式的定义及其代数精度的判别
%     \item Newton-Cotes求积公式的定义及几种常见的低阶公式(梯形公式、Simpson公式)的截断误差与代数精度
%     \item 复化梯形公式、复化Simpson公式应用及误差估计(先验与后验误差)
%     \item Romberg算法
%     \item Gauss型求积公式的定义及构造
% \end{enumerate}