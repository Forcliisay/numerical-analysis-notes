\chapter{绪论}
\section{误差}

\subsection{误差来源与分类}
\begin{enumerate}
    \item (\emph{模型误差}):从实际模型中抽象出数学模型;
    
    例如, 一个质量为$m$的小球做自由落体运动, 则位置$s$与时间$t$的关系式满足:
    \begin{equation*}
        m\dv[2]{s}{t}=mg
    \end{equation*}
    不难想见, 该式仅在\emph{不考虑阻力}时成立.

    \item (\emph{观测误差}):通过测量得到模型中参数的值;
    \item (\emph{方法误差}(或称\emph{截断误差})):求近似解时所引入的误差;
    \begin{example}
        考虑函数$f(x)$做Taylor多项式展开所导致的截断误差.
    \end{example}
    \begin{solution}
        对函数$f(x)$计算Taylor多项式, 有
        \begin{equation*}
            P_n(x)=f(0)+\frac{f'(0)}{1!}x+\frac{f''(0)}{2!}x^2+\cdots+\frac{f^{(n)}(0)}{n!}x^n
        \end{equation*}
        由于有限项, 因此多项式有截断误差
        \begin{equation*}
            R_n(x)=\frac{f^{(n+1)}(\xi)}{(n+1)!}x^{n+1}
        \end{equation*}
        其中, $\xi\in(x,0)$
    \end{solution}

    \item (\emph{舍入误差}):机器字长有限所引起的误差 
\end{enumerate}

其中, \emph{方法误差}和\emph{舍入误差}是数值分析所重点考虑的误差, 同时, \emph{方法误差是可以避免的}.

\subsection{误差概念}
\subsubsection{绝对误差与绝对误差限}
\begin{definition}[绝对误差与绝对误差限]
    设$x$是准确值, $x^*$是$x$的一个近似值, 则称
    \begin{equation*}
        e(x^*)=x^*-x
    \end{equation*}
    为$x^*$的\emph{绝对误差}, 简称\emph{误差}.

    同时, 误差的绝对值的上限$\varepsilon(x^*)$, 即有
    \begin{equation*}
        \abs{e(x^*)}=\abs{x^*-x}\le\varepsilon(x^*)
    \end{equation*}

    $\varepsilon(x^*)$称为\emph{绝对误差限}.
\end{definition}

\begin{notice}
    误差有正有负, 而误差限恒为正值.
\end{notice}

习惯上, 我们把精确值和测量值的关系表示为
\begin{equation*}
    x=x^*\pm\varepsilon
\end{equation*}

\subsubsection{相对误差与相对误差限}

\begin{definition}[相对误差与相对误差限]
    设$x$为准确值, $x^*$为近似值, 称
    \begin{equation*}
        e_r^*=e_r^*(x^*)=\frac{e(x^*)}{x}=\frac{x^*-x}{x}
    \end{equation*}
    为近似值$x^*$的\emph{相对误差}.

    同时, 其绝对值的上限$\varepsilon_r^*$, 即有
    \begin{equation*}
        \abs{\frac{x-x^*}{x}}\le \varepsilon_r^*
    \end{equation*}
    
    $\varepsilon_r^*$称为\emph{相对误差限}.
\end{definition}

可以证明, 当$e_r^*$较小时, 有
\begin{equation*}
    e_r^* \approx\frac{x^*-x}{x^*}
\end{equation*}
同时易得
\begin{equation*}
    \varepsilon_r^*=\frac{\varepsilon^*}{\abs{x^*}}
\end{equation*}

\subsubsection{有效数字}

\begin{definition}[有效数字, 有效位数, 有效数]
    若近似值$x^*$误差满足
    \begin{equation*}
        \abs{x-x^*}\le\frac{1}{2}\times10^{-n}
    \end{equation*}
    则称$x^*$近似表示$x$准确到小数点后第$n$位, 并从第$n$位起一直到最左边非零数字之间的一切数字称为\emph{有效数字}, 位数为\emph{有效位数}.

    若所有数字均为有效数字, 则称为\emph{有效数}
\end{definition}

\begin{example}
    考虑圆周率$\pi$, 且有近似值$\pi_1=3.14, \pi_2=3.1415, \pi_3=3.1416, \pi_4=3.14159$. 考虑它们的有效数字, 且判断是否为有效数.
\end{example}

\begin{solution}
    对于$\pi_1=3.14$, 有$\abs{\pi-\pi_1}\approx0.00159\le0.5\times10^{-2}$, 即$\pi_1$精确到小数点后2位, 有效数字是3位, 是有效数.

    同理, 有$\abs{\pi-\pi_2}\approx0.0000926\le0.5\times10^{-3}$, 即$\pi_2$精确到小数点后3位, 有效数字是4位, 不是有效数.

    $\abs{\pi-\pi_3}\approx0.0000073\le0.5\times10^{-4}$, 即$\pi_3$精确到小数点后4位, 有效数字是5位, 是有效数.

    $\abs{\pi-\pi_4}\approx0.0000026\le0.5\times10^{-5}$, 即$\pi_4$精确到小数点后5位, 有效数字是6位, 是有效数.
\end{solution}

从上例中不难看出, 有效数通常是采取\emph{四舍五入}所得到的近似值.

\begin{extend}
    我们可以简单给出关于\emph{四舍五入}的证明.

    \begin{proof}
        设准确值为$x$, 其近似值为$x^*$, 考虑近似值精确到小数点后$n$位, 即
        \begin{equation*}
            \abs{x-x^*}\le5\times10^{-(n+1)}
        \end{equation*}

        若其为有效数, 则$x^*$为小数点后$n$位, 不妨设
        \begin{equation*}
            x^*=a+b\cdot10^{-n}
        \end{equation*}
        其中$b\in[1,10)$

        特别地, 分两种情况讨论. 

        若$x>x^*$, 即真实值大于近似值, 此时有
        \begin{equation*}
            x\le x^*+5\times10^{-(n+1)}=a+b\cdot10^{-n}+5\times10^{-(n+1)}
        \end{equation*}
        即当小数点后第$n+1$位小于等于5时, 舍去后面的数字可以得到有效数.

        若$x<x^*$, 即真实值小于近似值, 此时有
        \begin{align*}
            x\ge x^*-5\times10^{-(n+1)}&=a+b\cdot10^{-n}-5\times10^{-(n+1)}\\
            &=a+(b-1)\cdot10^{-n}+5\times10^{-(n+1)}
        \end{align*}
        即当小数点后第$n+1$位大于等于5时, 进位可以得到有效数.
    \end{proof}
\end{extend}

